\documentclass{scrartcl}
\usepackage[latin1]{inputenc}
%\usepackage[T1]{fontenc}
\usepackage[ngerman]{babel}

\parindent0mm

\begin{document}

\titlehead{Kursleiterbericht Ro{\ss}leben 2017--5.2}
\subject{Bericht}
\title{Die Farbe Blau -- Spektroskopische Eigenschaften von chemischen Verbindungen}
\subtitle{Deutsche Sch{\"u}lerAkademie 2017 -- Ro{\ss}leben}
\author{Elke Fa{\ss}hauer und Mats Simmermacher}
\date{14.08.2017}
\maketitle

\section*{Kursverlauf}

In unserem Kurs \textit{Die Farbe Blau} haben wir uns mit der Farbigkeit chemischer Verbindungen befasst. Ausgehend von einfachen Atomen und den Grundlagen der Quantenmechanik haben die Teilnehmenden verschiedene Klassen blauer organischer und anorganischer Farbstoffe kennengelernt und sowohl deren Aufbau als auch die physikalischen Ursachen ihrer Farbe beschrieben. Von zentraler Bedeutung war dabei die Unterscheidung zwischen Beobachtung und Modellbildung. Durch zahlreiche Experimente konnten die Teilnehmenden eigene Beobachtungen machen, die es im Anschluss durch quantenchemische Erw{\"a}gungen und Computersimulationen zu erkl{\"a}ren und zu reflektieren galt. Durch dieses Wechselspiel experimenteller Praxis und theoretischer Beschreibung haben sich die Teilnehmenden die Grundprinzipien wissenschaftlichen Arbeitens und Denkens erschlossen. Zudem haben die Teilnehmenden zu Beginn eines jeden neuen Themenabschnitts Referate gehalten, deren Inhalt sie aus ihnen jeweils zugewiesener Literatur entnehmen konnten. So haben sie erste Erfahrungen im Umgang mit wissenschaftlichen Lehrb{\"u}chern und Publikationen sowie im Pr{\"a}sentieren wissenschaftlicher Zusammenh{\"a}nge machen k{\"o}nnen und waren gefordert, sich mit den Kursinhalten bereits im Vorfelde aktiv auseinander zu setzen. Die Teilnehmenden sollten dadurch zum eigenst{\"a}ndigen Arbeiten und zum kritischen Hinterfragen des von ihnen behandelten Stoffes animiert werden.\medskip

Grunds{\"a}tzlich haben wir unseren Kursplan gut einhalten k{\"o}nnen,
wenngleich sich die Wahl der Themen als etwas zu umfangreich herausgestellt hat.
Wir haben die verschiedenen Themen durch Referate der Teilnehmenden eingeleitet,
die wir h{\"a}ufiger selber widerholen mussten, da die
Teilnehmenden erhebliche Schwierigkeiten hatten, die wesentlichen Aspekte des
von ihnen zu pr{\"a}sentierenden Themas zu erkennen und einander zu
vermitteln.
Erschwerend kam hinzu, dass die Teilnehmenden deutlich weniger physikaffin
waren, als wir es uns bei der Konzeption des Kurses erhofft hatten.
W{\"a}hrend die vornehmlich chemischen Inhalte nur wenig Verst{\"a}ndnisprobleme
hervorgerufen haben und insgesamt sehr offen angenommen wurden, mussten wir die
quantenchemischen und physikalischen Elemente mehrfach wiederholen und
motivieren. Zwar war unser Kursplan flexibel genug, entsprechend darauf eingehen
zu k{\"o}nnen, doch haben wir aufgrund des Zeitverlustes die urspr{"u}nglich
angedachten Abschlussprojekte, in denen die Teilnehmenden kompliziertere
quantenchemische Computersimulationen durchf{\"u}hren sollten, streichen
m{\"u}ssen. Es ist wahrscheinlich, dass wir durch eine etwas engere Themenwahl
einige der Wiederholungen h{\"a}tten vermeiden und den urspr{\"u}nglichen
Kursplan besser h{\"a}tten einhalten k{\"o}nnen.
%So w{\"a}re es beispielsweise
%m{\"o}glich gewesen, statt der Phosphoreszenz die quantenmechanische Beschreibung
%des Wasserstoffatoms in einem Referat zu behandeln, wodurch die Teilnehmenden
%die Prinzipien der Quantenmechanik selbst ein weiteres Mal rekapituliert
%h{\"a}tten. Auch w{\"a}re den Teilnehmenden dadurch vermutlich schneller
%einsichtig gewesen, was Atomorbitale sind. Etwas ungl{\"u}cklich war auch der
%Umstand, dass wir das Problem des Teilchens im Kasten anhand einer einfachen
%Absch{\"a}tzung der Anregungsenergien in einem organischen Molek{\"u}l motiviert
%haben, bevor wir im Kurs {\"u}berhaupt auf Molek{\"u}le zu sprechen kamen. Wir
%h{\"a}tten daher bereits vor dem Teilchen im Kasten zumindest grob in die
%Molek{\"u}le einf{\"u}hren m{\"u}ssen, anstatt erst die kompletten
%quantenchemischen Grundlagen aufzubauen. Dadurch w{\"a}re den Teilnehmenden
%vermutlich bereits von Anfang an verst{\"a}ndlich geworden, warum die
%Quantenmechanik f{\"u}r die Chemie relevant und interessant ist, und wir
%h{\"a}tten weniger motivieren m{\"u}ssen.
\medskip

Zus{\"a}tzlich zu den Referaten, den Experimenten und den Computer{\"u}bungen
haben die Teilnehmenden Rechen{\"u}bungen auf dem Papier gel{\"o}st. Trotz der
anf{\"a}nglich leichten Aversion gegen{\"u}ber physikalischen Betrachtungen
sind die Teilnehmenden mit den Aufgaben sehr gut zurecht gekommen. Niemand hat
sich durch die {\"U}bungen {\"u}ber- oder unterfordert gef{\"u}hlt. Alle
Teilnehmenden konnten die gestellten Fragen am Ende angemessen beantworten und
Frustration kam nicht auf. F{\"u}r die mathematisch St{\"a}rksten im Kurs gab
es zudem zus{\"a}tzliche und optionale Knobelaufgaben, die von Einzelnen mit
gro{\ss}em Interesse behandelt wurden, auch wenn sie sie nicht immer
vollst{\"a}ndig zu l{\"o}sen wussten. Durch diese {\"U}bungen sind einige der
Grundlagen deutlich verfestigt worden.
%F{\"u}r die Nachbesprechung der Aufgaben
%haben wir allerdings etwas mehr Zeit in Anspruch genommen, als wirklich
%n{\"o}tig gewesen w{\"a}re. Wir h{\"a}tten an dieser Stelle gut eineinhalb Stunden einsparen k{\"o}nnen.
\medskip

Die Experimente kamen bei allen Teilnehmenden sehr gut an. Insbesondere der Luminolversuch hat die Teilnehmenden fasziniert. Ihr Verhalten im Labor war von einigen Kleinigkeiten abgesehen vorbildlich, sicher und zuverl{\"a}ssig. Die Versuchsdurchf{\"u}hrungen haben schnell zu den gew{\"u}nschten und f{\"u}r den Kursverlauf entscheidenden Ergebnissen gef{\"u}hrt. Allerdings hat der Luminolversuch etwas mehr Zeit in Anspruch genommen, als wir es uns vorab ausgemalt haben.
%Anstatt den Versuch in blo{\ss} einer Kurseinheit durchf{\"u}hren zu wollen, w{\"a}re es besser gewesen, ihn {\"u}ber zwei Einheiten zu verteilen. H{\"a}tten wir die Teilnehmenden bereits am Vormittag in den Versuch eingef{\"u}hrt und die verschiedenen Versuchsreihen erarbeiten lassen, h{\"a}tte den Teilnehmenden am Nachmittag mehr Zeit f{\"u}r die eigentliche Durchf{\"u}hrung des Versuches zur Verf{\"u}gung gestanden. Die Auswertung der Ergebnisse am Folgetag verlief daf{\"u}r wiederum sehr gut und z{\"u}gig.
\medskip

Auch die Computersimulationen sind von der Mehrheit der Teilnehmenden gut angenommen worden. Trotz anf{\"a}nglicher technischer Schwierigkeiten kamen die Teilnehmenden mit den an sie gestellten Aufgaben, den Anleitungen und dem Computerprogramm ADF gut zurecht. Gro{\ss}er Dank geb{\"u}hrt in diesem Zusammenhang
SCM (Software for Chemistry and Materials),
die uns f{\"u}r jeden Teilnehmenden kostenlos Lizenzen zur Verf{\"u}gung gestellt haben. Durch den Umgang mit ADF haben die Teilnehmenden sowohl ihr theoretisches Verst{\"a}ndnis der jeweils berechneten Verbindungen vertiefen als auch die Arbeitsweise der Computerchemie kennenlernen k{\"o}nnen. So sind sie mit einem selbst f{\"u}r viele Chemiestudierende weitgehend unbekannten Fach in Ber{\"u}hrung gekommen und haben ihren eigenen Forschungsdrang teilweise kultivieren und ausleben k{\"o}nnen. W{\"a}re am Ende noch ausreichend Zeit f{\"u}r das Abschlussprojekt vorhanden gewesen, h{\"a}tten die Teilnehmenden noch freier und eigenst{\"a}ndiger arbeiten und ihre Erfahrungen intensivieren k{\"o}nnen.\medskip

%An mehreren Stellen haben wir zudem methodische Einheiten eingeschoben. So haben wir mit unseren Teilnehmenden diskutiert, wodurch sich konstruktives Feedback auszeichnet, wie man es gibt und wie man mit Kritik umgeht, wenn man sie empf{\"a}ngt. In den sich an die Referate anschlie{\ss}enden Feedbackrunden konnten sich die Teilnehmenden dann in den erarbeitenen Grunds{\"a}tzen guter R{\"u}ckmeldung {\"u}ben. Auch haben wir die Teilnehmenden angeregt, {\"u}ber den Zweck einer Pr{\"a}sentation zu reflektieren, und besprochen, wie man gute und effektive Vortr{\"a}ge h{\"a}lt. Auch das konnte in den darauf folgenden Referaten erprobt werden.\medskip

Die Kursdurchmischung und der Zusammenhalt der Teilnehmenden untereinander war
hervorragend. Die Teilnehmenden haben sich sehr hilfsbereit und sozial
verhalten und sich sehr schnell als ein Team verstanden. Wir f{\"u}hren diesen
Umstand unter anderem darauf zur{\"u}ck, das wir der Bildung separater
Kleingruppen von Anfang an entgegen gewirkt haben. Vor jeder {\"U}bung mussten
sich die Teilnehmenden anhand von zuf{\"a}llig gezogenen Karten, auf denen
wir Kursinhalte zusammen gefasst haben und die einander zuzuordnen waren,
neu finden.\medskip

Zusammenfassend haben wir unsere Lernziele erreicht, sehen aber noch Verbesserungspotential in den Details der Kursdurchf{\"u}hrung.\bigskip

\section*{Technik und Material}

Das von uns im Vorfelde erbetene Material stand uns vollst{\"a}ndig zur Verf{\"u}gung. Wir haben alle Materialien verwendet und auch die Menge der Chemikalien war den Versuchen angemessen. Als sehr hilfreich hat sich in diesem Zusammenhang das Gespr{\"a}ch mit der Chemielehrerin der Klosterschule w{\"a}hrend des Vorbereitungstreffens erwiesen. So konnten wir in Erfahrung bringen, welche Ger{\"a}te und Chemikalien vor Ort vorhanden waren und von uns benutzt werden konnten, und welche wir extra bestellen mussten. Auch konnte die Frage der Entsorgung der Chemikalien mit der Schule abgesprochen werden. Wir legen der Kursleitung eines jeden zuk{\"u}nftigen Kurses mit experimentellen Inhalten in den Naturwissenschaften ausdr{\"u}cklich ans Herz, ein {\"a}hnliches Gespr{\"a}ch mit den verantwortlichen Lehrern zu suchen. Nat{\"u}rlich setzt das voraus, dass man sich bereits zum Vorbereitungstreffen dar{\"u}ber im Klaren ist, welche Versuche man durchzuf{\"u}hren gedenkt. Der Chemieleherin der Klosterschule Ro{\ss}leben m{\"o}chten wir zudem noch einmal unseren Dank aussprechen. Ihre Hilfsbereitschaft hat uns die Organisation der Experimente deutlich erleichtert.\medskip

Die acht Computer, die wir aus dem Fundus der Sch{\"u}lerAkademie erhalten haben,
haben grunds{\"a}tzlich gut funktioniert. Alle im Vorfelde erbetenen Programme
waren vorinstalliert. Der GitHub-Klient und das Programm ImageJay liefen ohne
jedwede Komplikation. Allerdings mussten wir \LaTeX auf f{\"u}nf der acht
Rechner neu installieren, da die entsprechenden Vorinstallationen nicht
funktionsf{\"a}hig waren. Durch die Verwendung der DSA-Formatvorlage f{\"u}r
die Dokumentation war zudem die Installation einiger Pakete erforderlich, die
sich in Anbetracht der eher schlechten Internetverbindung vor Ort als schwierig
erwies. Schlussendlich mussten wir die Pakete manuell kopieren und einbinden,
was sehr viel Zeit in Anspruch genommen hat, die uns im Kursverlauf eigentlich
nicht zur Verf{\"u}gung stand.
Im Endeffekt haben wir sp{\"a}t nachts die strategische Entscheidung getroffen,
dass Mats sich um die L{\"o}sung des Problems k{\"ü}mmern w{\"u}rde und Elke
schlafen ging, damit am n{\"a}chsten Tag eine Person in der Lage war, um den
Kurs zu halten.
Diese Mehrbelastung w{\"a}re vermeidbar gewesen, wenn die
\LaTeX -Installationen vorab auf ihre Funktionsf{\"a}higkeit {\"u}berpr{\"u}ft
worden w{\"a}ren. In Zukunft sollte insbesondere darauf geachtet werden, dass
alle f{\"u}r die DSA-Formatvorlage erforderlichen Pakete installiert sind. Es
w{\"a}re zudem hilfreich, wenn auch auf den Schulcomputern \LaTeX\ zur
Verf{\"u}gung st{\"u}nde. Weiter hat sich die von der Standard-\LaTeX -Syntax
abweichende Definition der DSA-Figure- und der DSA-Table-Umgebungen in der
Formatvorlage als verwirrend erwiesen, da die Teilnehmenden so nicht direkt von
oder auf die entsprechenden Standard-Befehle schlie{\ss}en konnten. Es w{\"a}re
in diesem Sinne hilfreich, wenn in Zukunft einfach die normalen Figure- und
Table-Umgebungen verwendet werden w{\"u}rden.\medskip

Ferner waren das im Kursraum vorhandene Smartboard und das von der Sch{\"u}lerAkademie bereitgestellte Geomag besonders hilfreich, um die Kursinhalte zu visualisieren.
Sollte ein Chemiekurs in Zukunft Experimente durchf{\"u}hren, in denen pipettiert werden muss, empfehlen wir, dass ein Satz frischer Peleusb{\"a}lle bestellt wird. Die in Schullaboratorien vorhandenen Peleusb{\"a}lle sind oft verunreinigt,
verklebt und aufgrund von Korrosion kaum zu gebrauchen.\medskip

Die Sammlung der Akademiebilder lief auf dieser Akademie etwas
unkoordiniert ab. In Zukunft sollte sich das AKL-Team im Vorwege auf eine
gemeinsame Strategie einigen, wie Photographien einfach gesammelt und
im Nachhinein verf{\"u}gbar
gemacht werden k{\"o}nnen.\bigskip

\section*{Standort}

Die Klosterschule Ro{\ss}leben hat unserer Auffassung nach als Akademiestandort {\"u}berzeugt. Die Kurs- und Chemier{\"a}ume waren f{\"u}r unsere Bed{\"u}rfnisse absolut geeignet. Als besonders positiv haben wir zudem das gro{\ss}e und im Internatsgeb{\"a}ude sehr zentral gelegene Lehrerzimmer, das wir als AL-B{\"u}ro nutzen konnten, die ebenfalls im Internatsgeb{\"a}ude befindliche Kirche, in der sowohl das Plenum als auch das Konzert und die entsprechenden Proben stattfanden, das gro{\ss}e, an die Schule angrenzende Parkgel{\"a}nde an der Unstrut sowie die Schulbibliothek mit teilweise 300 Jahre alten B{\"u}chern wahrgenommen. Die Geschichtslehrerin, die uns durch die Internatsgeschichte und die Bibliothek gef{\"u}hrt hat, hat sowohl uns als auch viele Teilnehmende stark und nachhaltig beeindruckt. Zuk{\"u}nftigen Akademien in Ro{\ss}leben m{\"o}chten wir an Herz legen, die Lehrerin erneut f{\"u}r eine F{\"u}hrung anzufragen. Sehr hilfreich war zudem die bereits oben erw{\"a}hnte Unterst{\"u}tzung und Hilfsbereitschaft der Chemielehrerin.\medskip

Lediglich die etwas kleine Turnhalle und die vergleichsweise abgelegene und
mit {\"o}ffentlichen Verkehrsmitteln schwer zu erreichende Lage der
Klosterschule sehen wir als Nachteile des Standortes an. F{\"u}r die Zukunft
m{\"o}chten wir weiter Folgendes empfehlen: Es sollte nach M{\"o}glichkeit
bereits auf dem Vorbereitungstreffen angesprochen werden, welche Fachsammlungen
im Schulgeb{\"a}ude zur Verf{\"u}gung stehen und welchen Zugang die einzelnen
Kurse ben{\"o}tigen. In unserem Falle w{\"a}re ein zus{\"a}tzlicher
Schl{\"u}ssel zur Physiksammlung beispielsweise praktisch gewesen. Da die
Schr{\"a}nke in der Sammlung alle aufgeschlossen waren, hatten wir dennoch
Zugriff auf den Inhalt der Schr{\"a}nke.\\
Der H{\"a}nger mit den in der Schule vorhandenen Kanus sollte w{\"a}hrend
der Exkursion nicht vom AKL-Team gefahren werden, da sich das Unterfangen als
sehr anstrengend und zeitraubend herausgestellt hat.
Es w{\"a}re besser, eine andere,
wom{\"o}glich externe L{\"o}sung zu finden.\\
Es ist zu {\"u}berlegen, ob die Mietautos statt in
Weimar besser in Erfurt abgegeben oder abgeholt werden sollten. Durch die
Stra{\ss}ensperrungen und die teilweise sehr engen und maroden
Landstra{\ss}en haben die Fahrten nach Weimar erhebliche Zeit in Anspruch
genommen. Um auf der Autobahn nach Weimar zu gelangen, muss man {\"u}ber
Erfurt fahren.\bigskip

\section*{Verpflegung}

Die Verpflegung in der Klosterschule war insgesamt gut, aber
verbesserungsf{\"a}hig. W{\"a}hrend das Fr{\"u}hst{\"u}ck und das Abendessen
sehr reichhaltig und vielf{\"a}ltig waren, war das vegetarische Mittagessen
oft nicht ausgewogen. Frisches Gem{\"u}se gab es zu selten. Gleichwohl hat
sich die K{\"u}che sichtlich bem{\"u}ht. Beispielsweise ist uns positiv
aufgefallen, dass in den Lunchpakaten zur Exkursion gelatinesfreies
Weingummi f{\"u}r die Vegetarier vorhanden war. In diesem Zusammenhang
w{\"u}rde uns das von Humangeographie-Kurs gef{\"u}hrte Interview mit dem Koch
sehr interessieren, der sich in der Dokumentation dieser Akademie befinden
wird.

\section*{AKL-Team}

Der Umgang und Zusammenhalt im AKL-Team war im Einzelnen sehr gut und harmonisch. Grunds{\"a}tzlich waren die Kursleitenden und auch die Akademieleitung sehr hilfsbereit und die Stimmung war gut. Besonders loben m{\"o}chten wir in diesem Zusammenhang den Einsatz Nikolais, die Schulcomputer einzurichten und f{\"u}r uns alle nutzbar zu machen. Allerdings fehlte unserem Eindrucke nach eine gemeinsame Zielsetzung, die bereits auf dem Vorbereitungstreffen h{\"a}tte erarbeitet werden k{\"o}nnen. So sind wir oft nicht wirklich als Team sondern mehr als nebeneinander agierende Kurs- und Akademieleitende aufgetreten. Wir sind der Auffassung, dass der Gruppenzusammenhalt mit einer gemeinsamen und verbindenen Vorstellung eines konkreten Akademieziels noch besser h{\"a}tte sein k{\"o}nnen. Das Team hatte in diesem Sinne mehr Potential.\medskip

Etwas ungl{\"u}cklich war der Besuch der Akademieleitung in unserem Kurs. Er kam sehr kurzfristig und zu einem eher unpassenden Zeitpunkt, als ein Kursteilnehmer gerade sein Referat hielt. Wir h{\"a}tten den Besuch gern besser in den Verlauf unseres Kurses eingebaut.\bigskip

\section*{Erste Hilfe}

Leider gab es nur wenig Ersthelfer im AKL-Team. Zudem war nicht ganz klar, wer im Falle eines Notfalls als Ersthelfer {\"u}berhaupt in Betracht kommt. In Zukunft sollte dem Thema Erste Hilfe wieder gr{\"o}{\ss}ere Aufmerksamkeit geschenkt werden. Insbesondere sollte auf dem Vorbereitungstreffen st{\"a}rker betont werden, dass der Abschluss eines Ersthelfer-Kurses ausdr{\"u}cklich begr{\"u}{\ss}t wird.\bigskip

\section*{Rotation}

Unsere Teilnehmenden haben sich bemerkenswert schnell in Gruppen f{\"u}r die Rotation zusammen gefunden. Insgesamt verlief die Zusammenarbeit in den Gruppen sehr gut, auch wenn einzelne, st{\"a}rkere Charaktere oft etwas dominiert haben. Schwieriger war hingegen die Findung eines geeigneten Themas f{\"u}r die Rotation. Die meisten Gruppen haben es sich zum Ziel gesetzt, einen m{\"o}glichst vollst{\"a}ndigen Eindruck unseres Kurses zu vermitteln. Anstatt sich auf ein einzelnes Thema zu beschr{\"a}nken, haben die meisten Gruppen mehrere Themen ausgew{\"a}hlt, die dann nur knapp behandelt werden konnten und deren {\"U}berg{\"a}nge f{\"u}r das Auditorium nicht immer nachvollziehbar waren. Die Schwierigkeiten, die viele unserer Teilnehmenden damit hatten, die wesentlichen Punkte ihres eigenen Referatsthemas herauszuarbeiten, wurden so auch in der Rotation offenbar.\medskip

W{\"a}hrend der Vorbereitung haben wir uns nur wenig eingemischt und den Teilnehmenden weitgehend freie Hand gelassen. Sie sollten ihre eigenen Erfahrungen machen k{\"o}nnen. Nach der Rotation haben wir das in der AKL-Teamsitzung emfangene Feedback in Einzelgesp{\"a}chen weitergegeben und den Teilnehmenden im Detail erkl{\"a}rt, was an ihrer Pr{\"a}sentation gut war und was sie wie verbessern sollten. Da die Selbsteinsch{\"a}tzung unserer Teilnehmenden oft positiver ausfiel als die des AKL-Teams, f{\"u}hrte unsere Kritik bei vielen zun{\"a}chst zu erkennbarer Frustration, wurde schlussendlich aber mit der Erkenntnis, viel {\"u}ber sich selbst gelernt zu haben, wohlwollend aufgenommen.\medskip

Die etwa 20 bis 25 Minuten, die jeder Gruppe im 6er-Schema der Rotation f{\"u}r ihre Pr{\"a}sentation zur Verf{\"u}gung standen, haben sich zudem als zu kurz erwiesen, um Kursinhalte gewinnbringend an Teilnehmende anderer Kurse vermitteln zu k{\"o}nnen. Das gilt insbesondere, da die entsprechenden Vorkenntnisse in Chemie und Physik nicht ohne Weiteres vorausgesetzt werden konnten und unsere Teilnehmenden so erst auf Grundlagen h{\"a}tten eingehen m{\"u}ssen. F{\"u}r die Zukunft pl{\"a}dieren wir daher wieder f{\"u}r das 4er-Schema, das den Teilnehmenden mehr Zeit f{\"u}r ihre eigenen Pr{\"a}sentationsbeitr{\"a}ge einr{\"a}umt.\bigskip

\section*{Dokumentation}

Die Dokumentation haben wir mit \LaTeX\ schreiben lassen. Da die Teilnehmenden keinerlei Vorkenntnisse bez{\"u}glich des Textsatzes mit \LaTeX\ hatten, haben wir die Teilnehmenden zun{\"a}chst in das Programm eingef{\"u}hrt. Trotz des Problems nicht funktionsf{\"a}higer Vorinstallationen haben unsere Teilnehmenden \LaTeX\ sehr bereitwillig und gut angenommen.\medskip

Bereits in der ersten Kurswoche haben wir auf die Dokumentation hingewiesen und immer wieder Kurseinheiten f{\"u}r das Verfassen von Dokumentationsbeitr{\"a}gen vorgesehen. Trotzdem haben viele Teilnehmende die Arbeit an der Dokumentation etwas schleifen lassen, sodass wir mehrfach deutlichen Ansto{\ss} geben mussten, mit der Arbeit zu beginnen oder fortzufahren. Die Qualit{\"a}t der zur Erstkorrektur eingereichten Entw{\"u}rfe war wie erwartet sehr heterogen. W{\"a}hrend einige Beitr{\"a}ge nur kleinere Verbesserungen erforderten, mussten andere Texte komplett umgeschrieben werden. Zu Beginn haben sich einige Teilnehmende zudem sehr schwer damit getan, unsere Korrekturanmerkungen umzusetzen. Auch die Kritik, die sich die Teilnehmenden untereinander gegeben haben, wurde nur selten direkt ber{\"u}cksichtigt. {\"U}ber die verschiedenen Korrekturzyklen hinweg konnten wir aber bei allen Teilnehmenden einen Lernprozess erkennen. Die Grunds{\"a}tze wissenschaftlichen Schreibens wurden zunehmend verinnerlicht und ber{\"u}cksichtigt und unsere Anmerkungen wurde immer besser eingearbeitet. Das betraf insbesondere das Setzen von Bildunterschriften und Literaturverweisen, aber auch die Grundstruktur der Texte und die wissenschaftliche Ausdrucksweise.\medskip

Als etwas ungl{\"u}cklich und st{\"o}rend empfanden wir es, dass parallel zu
der f{\"u}r alle vorgesehenen Dokumentationsarbeit am letzten Mittwoch-Morgen
Bandproben f{\"u}r das Konzert angesetzt wurden. So hatten wir nicht nur mit
L{\"a}rmbelastung zu k{\"a}mpfen, da der Bandraum sich unterhalb unseres
Kursraumes befand und der einzige trockene Zugang durch unseren Kursraum
f{\"u}hrte, sondern auch die Abwesenheit einiger
Teilnehmender zu beklagen, die eigentlich an ihren Texten h{\"a}tten arbeiten
m{\"u}ssen und stattdessen der Musik nachgingen. Das hat den Zeitdruck
unn{\"o}tig verst{\"a}rkt und zus{\"a}tzlich Kraft und Konzentration
gekostet.\medskip

Durch das Feedback der Teilnehmenden in der letzten Kurseinheit erfuhren wir dessen ungeachtet, dass die Arbeit an der Dokumentation als besonders lehr- und hilfreich wahrgenommen wurde.

\end{document}
